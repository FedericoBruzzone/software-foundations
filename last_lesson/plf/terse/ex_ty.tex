\documentclass[12pt]{report}
\usepackage[]{inputenc}
\usepackage[T1]{fontenc}
\usepackage{fullpage}
\usepackage{coqdoc}
\usepackage{amsmath,amssymb}
\usepackage{url}
\begin{document}
%%%%%%%%%%%%%%%%%%%%%%%%%%%%%%%%%%%%%%%%%%%%%%%%%%%%%%%%%%%%%%%%%
%% This file has been automatically generated with the command
%% coqdoc --gallina --latex ex_ty.v 
%%%%%%%%%%%%%%%%%%%%%%%%%%%%%%%%%%%%%%%%%%%%%%%%%%%%%%%%%%%%%%%%%
\coqlibrary{ex ty}{Library }{ex\_ty}

\begin{coqdoccode}
\coqdocnoindent
\coqdockw{From} \coqdocvar{Coq} \coqdockw{Require} \coqdockw{Import} \coqdocvar{Arith.Arith}.\coqdoceol
\coqdocnoindent
\coqdockw{From} \coqdocvar{PLF} \coqdockw{Require} \coqdockw{Import} \coqdocvar{SmallstepAM}.\coqdoceol
\coqdocnoindent
\coqdockw{From} \coqdocvar{PLF} \coqdockw{Require} \coqdockw{Import} \coqdocvar{TypesAM}.\coqdoceol
\coqdocnoindent
\coqdockw{From} \coqdocvar{Hammer} \coqdockw{Require} \coqdockw{Import} \coqdocvar{Tactics}.\coqdoceol
\coqdocnoindent
\coqdockw{Import} \coqdocvar{TM}.\coqdoceol
\coqdocemptyline
\end{coqdoccode}
In this exercise, we explore different ways to talk about the
arithmetic language seen in Types. You can use \coqdocvar{sauto}, but you *must* pass it
the right arguments (ctrs/inv/use)



Part 1.1: define a relational big step semantics for evaluating  \coqdocvar{tm}.

\begin{coqdoccode}
\coqdocemptyline
\coqdocnoindent
\coqdockw{Inductive} \coqdoctac{eval} : \coqdocvar{tm} \ensuremath{\rightarrow} \coqdocvar{tm} \ensuremath{\rightarrow} \coqdockw{Prop} :=.\coqdoceol
\coqdocemptyline
\end{coqdoccode}
Part 1.2. now prove the value soundness theorem:



\begin{coqdoccode}
\coqdocnoindent
\coqdockw{Theorem} \coqdocvar{vs}: \coqdockw{\ensuremath{\forall}} \coqdocvar{t} \coqdocvar{v}, \coqdoctac{eval} \coqdocvar{t} \coqdocvar{v} \ensuremath{\rightarrow} \coqdocvar{value} \coqdocvar{v}.\coqdoceol
\coqdocemptyline
\end{coqdoccode}
If you have problems, perhaps
you need a different notion of \coqdocvar{value}

 1.3. Show that \coqdoctac{eval} is a \coqdocvar{partial}  function
\begin{coqdoccode}
\coqdocemptyline
\coqdocnoindent
\coqdockw{Theorem} \coqdocvar{eval\_det}: \coqdocvar{deterministic} \coqdoctac{eval}.\coqdoceol
\coqdocemptyline
\end{coqdoccode}
1.4. Prove preservation: 
\begin{coqdoccode}
\coqdocemptyline
\coqdocnoindent
\coqdockw{Theorem} \coqdocvar{preservationB} : \coqdockw{\ensuremath{\forall}} (\coqdocvar{t}  : \coqdocvar{tm}) \coqdocvar{T},\coqdoceol
\coqdocindent{1.00em}
\coqdocvar{has\_type} \coqdocvar{t} \coqdocvar{T} \ensuremath{\rightarrow} \coqdockw{\ensuremath{\forall}} \coqdocvar{t'},\coqdoceol
\coqdocindent{1.00em}
\coqdoctac{eval} \coqdocvar{t} \coqdocvar{t'} \ensuremath{\rightarrow}\coqdoceol
\coqdocindent{1.00em}
\coqdocvar{has\_type} \coqdocvar{t} \coqdocvar{T}.\coqdoceol
\coqdocemptyline
\end{coqdoccode}
Part 2: write a ( functional) type checker. You will need to define
a \textit{boolean} test for \coqdocvar{ty} equality. To make the code more concise, we suggest
 to use the let monadic notation used in the ImpCevalFun chapter. 
\begin{coqdoccode}
\coqdocemptyline
\coqdocindent{0.50em}
\coqdockw{Fixpoint}  \coqdocvar{typeof}\coqdoceol
\coqdocindent{2.50em}
(\coqdocvar{t} : \coqdocvar{tm}) :  \coqdocvar{option} \coqdocvar{ty}.\coqdoceol
\coqdocindent{0.50em}
\coqdocvar{Admitted}.\coqdoceol
\coqdocemptyline
\end{coqdoccode}
2.1 Prove that the relational version entails the functional one. 
\begin{coqdoccode}
\coqdocemptyline
\coqdocindent{0.50em}
\coqdockw{Theorem} \coqdocvar{rel2f}: \coqdockw{\ensuremath{\forall}} \coqdocvar{t} \coqdocvar{T}, \coqdocvar{has\_type} \coqdocvar{t} \coqdocvar{T} \ensuremath{\rightarrow} \coqdocvar{typeof} \coqdocvar{t} = \coqdocvar{Some} \coqdocvar{T}.\coqdoceol
\coqdocemptyline
\end{coqdoccode}
Extra credit: prove the other direction
\begin{coqdoccode}
\coqdocemptyline
\coqdocemptyline
\end{coqdoccode}
\end{document}
